\documentclass[12pt]{article}

\usepackage[hidelinks]{hyperref}
\usepackage{graphicx}
\usepackage{float}
\usepackage[backend=biber]{biblatex}
\addbibresource{bibliography-file/document-mastery.bib}

\title{Mastering Documentation: The Road to a Documentation System}
\author{Mario Fernández}
\date{\today}

\begin{document}
\maketitle
\newpage
\tableofcontents
\newpage

\section{Introduction}

\subsection{Humble Beginnings}
This project started out from a YouTube video\cite{miasdigitaldiaryHowChangeYour2025}, in which the author exposes her approach to experimenting as a way to expand our own limitations when it comes to learning and work. This deeply inspired me back in December of 2025, so I decided to expand on these themes in 2026 as a way to change my life with structure and meaning, ridding myself form mediocracy as a Software Engineer by learning new skills and all the random and weird techonologies and fields I've always felt drawn to during college. For each month, I would experiment in a new area or topic that I loved, without \emph{any} type of rules, predictions, judgements or blame: just pure ``just do it''\cite{nikeincNikeJustIt2026} energy, for the sake of learning and doing something useful using structured guidelines. There wouldn't be a way forward if I didn't set this for myself. So, here we are!

\subsection{What is this about?}
This document sets to demonstrate a documentation system in which I've been working and training myself to use during January. This documentation system aims to structure everything I do , I use and I store related to documentation, therefore --- \textbf{documentation mastery}! I proposed myself to learn all these tools that I considered \textit{vital} in my documentation journey, to at least a medium level, mastering main aspects and workflows. I honestly feel very good doing this, I've expanded myself a lot and I feel proud of myself.

\subsection{Document Tool Kit}
This document you're reading right now is a demonstration of the documentation system working in perfect unison. I won't spoil ahead of myself but all of the layers of the system were involved in the creating and publication of this document. You're reading this because of my hope for inspiration to strike you in your time here. Hope you get something working for yourself and for future you, something that can really contribute to your creative areas and your profession, just like I did with myself. But without any more hesitation, let's dive into it!

\newpage

\section{Documentation System}
\subsection{What do you mean by \textit{Documentation System}?}
When we discuss documentation, an important fact comes to notice: documentation is the art of storing information. Information, also known as (aka) data, is directly involved in documentation, in it's inputs and outputs. This is the main objective of the documentation system: to spread to early and final stages of documentation and organize them, from when the document is just mere influences and when it's on the last writing phase. Now, in relation to understanding the naming, let's firstly understand what a system is: as explained on the Merriam-Webster dictionary, specifically on definition 3 \cite{DefinitionSYSTEM2026}, a system is an organized or established procedure. In relation to documents, it would stand for an organized or established procedure of handling documentation. Therefore, a documentation system is a series of steps or actions that involve activities and tools in order to work with documentation efficiently.

\subsection{Key Areas of the System}
The system would encompass the main aspects of my documentation process,  starting with the capture of information. This was essential to me personally, as I've been struggling with it for most of my life. I always forgot to save my ideas or the inspiration that generated them. Different years, different tools, same problem. It was clear that I needed some well-thought structure. The other area I was struggling the most had to do with transforming that information. I didn't have a place where I could go out and about with those little inspirations, and in consequence I switched apps several times without feeling fully at home.

These obstacles and all the environment changes refined my wants and needs. College, along with Computer Science, broadened my understanding even further. I established \textbf{three main stages} for the system:

\begin{figure}[H]
\caption{Three main stages of the documentation system.}
\label{fig:main-stages}
\centering
\includegraphics[height=0.4\textheight]{figures/main-processes.png}
\end{figure}

As shown in Figure 1, the documentation system is divided in three different processes, each of them covering a major area when dealing with documentation and information. In order and in a more detailed glance:

\begin{enumerate}
\item{Capture: Capturing refers to the input of information, regardless of it's form. It's goal is to safely and accurately store documents and other media for longetivity.}
\item{Processing: It's the moment where those ideas and thoughts previously captured find their way into my own documents and notes. It must include a note-taking app to give myself some liberty.}
\item{Expression: This is where a new document comes to life. It stands upon the other areas, so they must be a solid base for success. It must include a word processor software.}
\end{enumerate}

Along the way, I found several new areas to be optimized as well, as a part of the documentation system. These areas live inside the Processing stage and are important to the vitality and productivity of the workflow. After some analyzing, I expanded the system to include \textbf{two more areas} inside stage two.

\begin{figure}[H]
\caption{Two sub-areas are incorporated to the documentation system.}
\label{fig:main-stages}
\centering
\includegraphics[height=0.4\textheight]{figures/all-processes.png}
\end{figure}

These areas were \textit{vital} during my previous experiences. I wanted to incorporate those potential flaws I already went through in the past. Here are the details:

\begin{itemize}
\item{Historical Record: What happens if by accident I delete a very important file? Or during rewriting I get rid of a part of an investigation that, now two months later, I need? Guardrails must be put in place to avoid loss of information and to have a backup of some sort.}
\item{Editing Experience: I need something nice and comfortable. I go very fast with my keyboard, but no mainstream programs use the keyboard only. The switching between the mouse and the keyboard will worsen the experience.}
\end{itemize}

After setting these as my focus, I started investigating new tools, reinstalling old ones, and comparing various software applications. There were a lot of options, but I believe I settled for the right ones for the job. 

\newpage

\section{Deep System Analysis}
\subsection{Capture --- Zotero}
\subsection{Processing --- Obsidian}
\subsection{Historical Record --- Git}
\subsection{Editing Experience --- Vim}
\subsection{Expression --- \LaTeX}

\newpage

\printbibliography[title=References]

\end{document}
