\documentclass[12pt]{article}

\usepackage[hidelinks]{hyperref}
\usepackage{graphicx}
\usepackage{float}
\usepackage[backend=biber]{biblatex}
\addbibresource{bibliography-file/document-mastery.bib}

\title{Mastering Documentation: The Road to a Documentation System}
\author{Mario Fernández}
\date{\today}
\pagenumbering{roman}
\begin{document}
\maketitle
\newpage

\tableofcontents
\newpage
\pagenumbering{arabic}
\section{Introduction}

\subsection{Humble Beginnings}
While I was enjoying YouTube videos, I came across a video about how to change your life in 2026\cite{miasdigitaldiaryHowChangeYour2025} where the author exposes her approach to experimenting as a way to expand our own limitations when it comes to learning and work. She advices to use an `experimental mindset' as a way to learn and dive into new things, where ``even failure is outcome''. This deeply inspired me back in December of 2025, where I took it as divine timing and a confirmation to take action, so I decided to incorporate it into my new year as a way to change my life with structure and meaning, ridding myself form mediocrity as a Software Engineer by learning new skills and all the random and weird techonologies and fields I've always felt drawn to. For each month, I would experiment in a new area or topic that I loved, without \emph{any} type of rules, predictions, judgements or blame: just pure ``just do it''\cite{nikeincNikeJustIt2026} energy, for the sake of learning and doing something useful using structured guidelines. There wouldn't be a way forward if I didn't set this for myself. So, here we are!

\subsection{What is this about?}
This document sets to demonstrate a documentation system in which I've been working and training myself to use during January. This documentation system aims to structure everything I do , I use and I store related to documentation, therefore --- \textbf{documentation mastery}! I proposed myself to learn all these tools that I considered \textit{vital} in my documentation journey, to at least a medium level, mastering main aspects and workflows. I honestly feel very good doing this, I've expanded myself a lot and I feel proud of myself.

\subsection{Document Tool Kit}
This document you're reading right now is a demonstration of the documentation system working in perfect unison. I won't spoil ahead of myself but all of the layers of the system were involved in the creating and publication of this document. You're reading this because of my hope for inspiration to strike you in your time here. Hope you get something working for yourself and for future you, something that can really contribute to your creative areas and your profession, just like I did with myself. But without any more hesitation, let's dive into it!

\newpage

\section{Documentation System}
\subsection{What do you mean by \textit{Documentation System}?}
When we discuss documentation, an important fact comes to notice: documentation is the art of storing information. Information, also known as (aka) data, is directly involved in documentation, in it's inputs and outputs. This is the main objective of the documentation system: to spread to early and final stages of documentation and organize them, from when the document is just mere influences and when it's on the last writing phase. Now, in relation to understanding the naming, let's firstly understand what a system is: as explained on the Merriam-Webster dictionary, specifically on definition 3 \cite{DefinitionSYSTEM2026}, a system is an organized or established procedure. In relation to documents, it would stand for an organized or established procedure of handling documentation. Therefore, a documentation system is a series of steps or actions that involve activities and tools in order to work with documentation efficiently.

\subsection{Key Areas of the System}
The system would encompass the main aspects of my documentation process,  starting with the capture of information. This was essential to me personally, as I've been struggling with it for most of my life. I always forgot to save my ideas or the inspiration that generated them. Different years, different tools, same problem. It was clear that I needed some well-thought structure. The other area I was struggling the most had to do with transforming that information. I didn't have a place where I could go out and about with those little inspirations, and in consequence I switched apps several times without feeling fully at home.

These obstacles and all the environment changes refined my wants and needs. College, along with Computer Science, broadened my understanding even further. I established \textbf{three main stages} for the system:

\begin{figure}[H]
\caption{Three main stages of the documentation system.}
\label{fig:main-stages}
\centering
\includegraphics[height=0.4\textheight]{figures/main-processes.png}
\end{figure}

As shown in Figure 1, the documentation system is divided in three different processes, each of them covering a major area when dealing with documentation and information. In order and in a more detailed glance:

\begin{enumerate}
\item{Capture: Capturing refers to the input of information, regardless of it's form. It's goal is to safely and accurately store documents and other media for longetivity.}
\item{Processing: It's the moment where those ideas and thoughts previously captured find their way into my own documents and notes. It must include a note-taking app to give myself some liberty.}
\item{Expression: This is where a new document comes to life. It stands upon the other areas, so they must be a solid base for success. It must include a word processor software.}
\end{enumerate}

Along the way, I found several new areas to be optimized as well, as a part of the documentation system. These areas live inside both the Processing stage and the Expression stage, and are important to the vitality and productivity of the workflow. After some analyzing, I expanded the system to include \textbf{two more areas} inside stage two.

\begin{figure}[H]
\caption{Two sub-areas are incorporated to the documentation system.}
\label{fig:main-stages}
\centering
\includegraphics[height=0.4\textheight]{figures/all-processes.png}
\end{figure}

These areas were \textit{vital} during my previous experiences. I wanted to incorporate those potential flaws I already went through in the past. Here are the details:

\begin{itemize}
\item{Historical Record: What happens if by accident I delete a very important file? Or during rewriting I get rid of a part of an investigation that, now two months later, I need? Guardrails must be put in place to avoid loss of information and to have a backup of some sort.}
\item{Editing Experience: I need something nice and comfortable. I go very fast with my keyboard, but no mainstream programs use the keyboard only. The switching between the mouse and the keyboard will worsen the experience.}
\end{itemize}

After setting these as my focus, I started investigating new tools, reinstalling old ones, and comparing various software applications. There were a lot of options, but I believe I settled for the right ones for the job. 

\newpage

\section{Deep System Analysis}
\subsection{System Layers}
\subsubsection{Capture --- Zotero}
Capturing information and documents that inspire thought and actions is managed by Zotero. Zotero is a tool to collect, organize, annotate, cite and share research \cite{zoteroZoteroYourPersonal2025}.  This allows me to store information without caring the form it's presented. It's the best way to store all those random sparks of creativity I get while watching youTube videos along with it's sources. It also has mobile applications, so I can store an article I'm reading on the bus while away from my desktop computer. Another advantage is the price: it's free of use! There's obviously paid plans but they only offer storage space in the cloud, which I won't be needing as much, so the choice is clear. This way I'm training myself for a future research era.
\subsubsection{Processing --- Obsidian}
The information needs to be modified, refined, molded. That's what Obsidian is for. Obsidian is an offline note taking app that uses open formats and offers connections between your notes \cite{obsidianObsidianSharpenYour2025}. It functions like a human brain: storing links between notes and themes, so it makes an easier way of recollecting information and analysing common themes. It's extensive list of plugins make \textit{almost} anything possible within the app. You name it: database queries of your notes, podcasts integrated with notes, soundscapes to listen to music while in the app, a drawing plugin (the one used for the diagrams in this document), infinite canvases to lay out notes and images, and a whole lot more. It's a space where I can write, edit and modify as much as I want, expanding on concepts and themes as I please. Want to be curious about turtles and how they go to the beach to lay their eggs? Written, made a couple of related notes and drove conclusions. The only limitation is your own creativity and imagination. I fully recommend this application. It's made my life better, even after spending years ignoring it. I'm actually paying for the Sync function, which allows me to have my files in all my devices. It's super intuitive and automatic, which is why I chose the platform's service, though there's plugins integrating Google Drive, Dropbox and other services for the same goal. You set the limit and the ecosystem where you'll work. Truly amazing.
\subsubsection{Historical Record --- Git}
In order to have a backup of my files, including data such as dates, times, file routes; I knew I needed a version control system. I learned about the concept during my academic years in college, so I knew what I wanted to use: Git. Git is a free and open source verson control system designed to work fast and efficiently without caring for sizes \cite{gitcommunityGit2026}. It would act in all other areas possible, storing my progress in my work either to be kept locally, to collaborate with others on the internet or to be published on GitHub. The fact that is widely used as a standard made the choice much easier. With Git covering my back, I'll have backups in place in case I decide to turn my manuscript back in time to an earlier version or if I don't like a change I did with my markdown files inside Obsidian and have to revert to a different version stored in a different branch. Pure security bliss.
\subsubsection{Editing Experience --- Vim}
I wanted to be productive while editing files and working with the keyboard. I frequently lose time getting the mouse to change the location of the cursor to keep writing somewhere else. It's the same when I'm writing code or I'm using the terminal: there's something that's missing. During my YouTubing time I found an engineer that worked as a senior in Netflix, and in some of his videos he mentions this program called Vim, which was created in the 90s and made him gain in speed and automation because of Vim Motions: quick commands that allow to delete a set amount of words pressing a couple of keys, moving fluently through documents, incorporating macros to automatize editing, all while keeping my hands on the keyboard as I usually do! I knew I needed to upgrade in this way, it would return in too many upskills. After some research, the way was using a modern fork: Neovim, a spiritual sucessor that has plugin integration and it's still being maintained \cite{neovimteamNeovim}. It makes me go at the speed of light and accomodates my workflow and concentration. And it works to any type of file: I can code from the same program and use the terminal as well, all while having the motions with me. A new level.
\subsubsection{Expression --- \LaTeX}
I've been writing documents for the majority of my life, mainly for school homework and assignments. I got very good at using Microsoft Word during college, learning all the little details and quirks that nobody had the curiosity to understand. But when my bachelor thesis was getting close, I heard of a technology the graduates were forced to use to write their thesis in and that most of them thought it was `too hard' to use: \LaTeX. Something inside moved me to investigate it, and I was totally right. \LaTeX \space is a document preparation system that is designed for the production of technical and scientific documentation \cite{thelatexprojectLaTeXDocumentPreparation2026}. It works with code instead of interfaces: you basically have source code that you type along with commands to then generate the document from it. I knew this was superior to any other tool I've typed in. I have so much control over normally-ignored details in the document, and I also have liberties I usually didn't have, like not caring about sorted lists, the numbers classify and organize themselves here. Another strong point for \LaTeX \space is the mathematic capacities. As a math boy I've always loved formulas and equations, and \LaTeX \space exceeds at it, being basically a standard. i want to demonstrate it with the Relativity equation:

\begin{equation}
E = mc^2
\end{equation} 

It's automatic and natural as a Computer Science engineer. It's all strong points. This is perfect to materialize and express my ideas and information into the form of documents, the most complete form. Out of this world.

\subsection{System Integration}
it's all thought to be interconnected. Zotero stores information, which gets developed inside Obsidian with Git's backup and Vim's editing speed, all to go into a \LaTeX \space document. But at the same time, Git keeps track of the \LaTeX \space files I'm working on, while Vim can edit at amazing speeds the source code of the document. It's all focused towards the decrease of stress and dread while working on documents. And it feels like the objective has been completed.

\newpage

\section{Closing Thoughts}
This system demonstrates that change and improvement is possible. I did not thought that I would be here writing this document after designing and connecting multiple new tools that I never thought of using, and yet here we are, and I'm freaking enjoying it. Always strike for better, always go chasing the things you love and have pure curiosity about. Don't let anyone, and I really mean ANYONE convince you to do otherwise. Even if you're the only one in your whole city (chances are I'm the only one in my city LOL), go do it, go accomodate your life the way you want. This is testimony that it's possible. I hope this inspired you somehow, go do you, go pay tribute to your authenticity, because you, and most importantly the world, needs you.

\newpage

\printbibliography[title=References]

\end{document}
