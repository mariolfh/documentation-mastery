\documentclass[12pt]{article}

\usepackage[hidelinks]{hyperref}
\usepackage{graphicx}
\usepackage[backend=biber]{biblatex}
\addbibresource{bibliography-file/document-mastery.bib}

\title{Mastering Documentation: The Road to a Documentation System}
\author{Mario Fernández}
\date{\today}

\begin{document}
\maketitle
\newpage
\tableofcontents
\newpage

\section{Introduction}

\subsection{Humble Beginnings}
This project started out from a YouTube video\cite{miasdigitaldiaryHowChangeYour2025}, in which the author exposes her approach to experimenting as a way to expand our own limitations when it comes to learning and work. This deeply inspired me back in December of 2025, so I decided to expand on these themes in 2026 as a way to change my life with structure and meaning, ridding myself form mediocracy as a Software Engineer by learning new skills and all the random and weird techonologies and fields I've always felt drawn to during college. For each month, I would experiment in a new area or topic that I loved, without \emph{any} type of rules, predictions, judgements or blame: just pure ``just do it''\cite{nikeincNikeJustIt2026} energy, for the sake of learning and doing something useful using structured guidelines. There wouldn't be a way forward if I didn't set this for myself. So, here we are!

\subsection{What is this about?}
This document sets to demonstrate a documentation system in which I've been working and training myself to use during January. This documentation system aims to structure everything I do , I use and I store related to documentation, therefore --- \textbf{documentation mastery}! I proposed myself to learn all these tools that I considered \textit{vital} in my documentation journey, to at least a medium level, mastering main aspects and workflows. I honestly feel very good doing this, I've expanded myself a lot and I feel proud of myself. But without any more hesitation, let's dive into it!

\newpage

\section{Documentation System}
\subsection{What do you mean by \textit{Documentation System}?}
When we discuss documentation, an important fact comes to notice: documentation is the art of storing information. Information, also known as (aka) data, is directly involved in documentation, in it's inputs and outputs. This is the main objective of the documentation system: to spread to early and final stages of documentation and organize them, from when the document is just mere influences and when it's on the last writing phase. Now, in relation to understanding the naming, let's firstly understand what a system is: as explained on the Merriam-Webster dictionary, specifically on definition 3 \cite{DefinitionSYSTEM2026}, a system is an organized or established procedure. In relation to documents, it would stand for an organized or established procedure of handling documentation. Therefore, a documentation system is a series of steps or actions that involve activities and tools in order to work with documentation efficiently.

\section{Key Areas of the System}
The system would encompass the main aspects of my documentation process,  starting with the capture of information. This was essential to me personally, as I've been struggling with it for most of my life. I always forgot to save my ideas or the inspiration that generated them. Different years, different tools, same problem. It was clear that I needed some well-thought structure. THe other area I was struggling the most had to do with transforming that information. I didn't have a place where I could go out and about with those little inspirations, and in consequence I switched apps several times without feeling fully at home.

These obstacles and all the environment changes refined my wants and needs. College, along with Computer Science, broadened my understanding even further. I established \textbf{three main stages} for the system, shown below.

\includegraphics[height=0.3\textheight]{figures/main-processes.png}



\newpage

\printbibliography[title=References]

\end{document}
